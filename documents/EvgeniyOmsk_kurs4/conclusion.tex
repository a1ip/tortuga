\chapter*{Заключение}						% Заголовок
\addcontentsline{toc}{chapter}{Заключение}	% Добавляем его в оглавление

\textbf{Достигнутые результаты:}
\begin{itemize}
  \item Добавлены функциональные возможности:
	\begin{itemize}
	  \item очистки экрана
	  \item изменения толщины рисования
	  \item изменения вида концов линий
	  \item изменения координат и угла поворота
	\end{itemize}
  \item Добавлены плагины:
  	\begin{itemize}
  	  \item сокращения ссылок
  	  \item перестроения урока без перезагрузки
  	\end{itemize}
  \item Написаны модули:
  	\begin{itemize}
  	  \item редактирования уроков
  	  \item обработки событий мыши
  	  \item обработки перемещенных файлов
  	\end{itemize}
  \item Реализована автоматизация сборки
  \item Произведен рефакторинг архитектуры
\end{itemize}
\vspace{10mm}

Web-приложение «Тортуга» создано для обучения школьников основам программирования, и для достижения максимального результата этот процесс должен быть удобен как ученику, при выполнении заданий, так и учителю, при формировании урока. Длинные ссылки вида (2), которыми затруднительно делиться, использовать в презентациях, размещать в соц-сетях, отсутствие инструмента для редактирования готовых уроков, отсутствие функции очистки экрана и изменения координат и угла поворота являлись слабым местом приложения, требующие исправления.\par

Что бы приложение могло развиваться и дальше, его необходимо пересмотреть и реорганизовать. Удалить взаимозависимость между компонентами, так что бы изменение в одном компоненте не влекло изменения в других. А от определения шаблона проектирования зависит не только решение поставленных задач, но и видение дальнейших шагов для достижения поставленной цели.\par

Возможность  автоматизации сборки веб-приложения обладает самым крупным потенциальным резервом для повышения эффективности разработки, снижения требуемых материальных и трудовых ресурсов, сокращения монотонной работы, повышения производительности разработчиков и качества выпускаемого продукта. Тем самым ускорив его разработку.\par




\clearpage