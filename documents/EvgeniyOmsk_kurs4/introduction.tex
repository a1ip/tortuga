Для обучения детей младшего и среднего возраста широко применяются графические исполнители. Программист при помощи команд управления неким роботом, который может передвигаться по плоскости и чертить на ней рисунки. Для этого используется простой язык программирования, включающий систему команд непосредственного управления роботом, т.е. команды перемещения и рисования, а также управляющие конструкции, которые позволяют организовать повторения, ветвления, выделить какие–то действия в подпрограмму или процедуру.\par
В качестве такого языка часто используется Logo~\cite{shaposhnicow}. В Омской физико–математической школе № 64 в 90-е годы прошлого века в течение долгого времени использовалась программа MSW Logo от Softronics Home Page ~\cite{logo}. К сожалению развитие этого продукта остановилось, и даже интерфейс продукта выглядит несколько инородно в современных версиях OS Windows. На данный момент лицей № 64 уже давно не использует этот продукт. Сейчас там используется Scratch-проект Массачусетского технологического института ~\cite{scratch}.\par
ООО «Образование IT» одна из омских организаций, которая занимается обучением школьников программированию в рамках проекта «Школа программиста». Её не устроили известные графические исполнители, реализованные в виде десктопных приложений, в силу громоздкости интерфейса, сложности установки, ограниченности поддерживаемых платформ. С одной стороны, хотелось иметь продукт, который бы не требовал установки вовсе, или устанавливался бы очень легко, а так же поддерживался всеми популярными платформами: Windows, OS X, Linux. Выбор упал на web-приложение, ведь для его работы необходим только современный браузер, который с большой вероятностью используется на компьютере в учебных заведениях и дома у учеников. С другой стороны, хотелось получить удобство и гибкость в создании и поддержке методических материалов, а так же иметь независимость от наличия интернета, чтобы иметь возможность заниматься в летних выездных школах в оздоровительных лагерях, а так же в рядовых городских образовательных заведениях, где перебои с интернетом-вещь весьма обычная. Поэтому «Школу программиста» не устроили существующие, на данный момент, Web аналоги ~\cite{blockly,codecademy,kodu,appinventor,alice,onlibelogo,logointerpr,logotortoise,tortuelogo,papert,}, которые не дают удобной возможности организовывать библиотеку методических материалов, а так же, как правило, требуют наличие соединения с сервером. Поэтому было принято решение создать собственного графического исполнителя, который обладал бы всеми необходимыми свойствами: работал бы в браузере, позволял организовывать методические материалы, не был бы привязан к интернету. Так появился проект «Тортуга».\par
«Тортуга» уже использовался для обучения школьников в течение весны 2013 года. В рамках этого проекта передо мной стояла задача по расширению функциональных возможностей программы. Описанные в данной работе результаты на данный момент включены в релизную версию программы, которая использовалась преподавателями «Школы программиста» в летних выездных и при школьных лагерях в июне 2013 года.


\clearpage