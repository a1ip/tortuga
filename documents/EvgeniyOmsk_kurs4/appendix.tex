\appendix
\chapter{} 

\textbf{\Large A1. Сокращение ссылки} \label{AppendixA3}
\vspace{6mm}

Отправляет запрос с передачей длинной ссылки:
\begin{verbatim}
var parseShortenedResponse = function(m, longUrl) {
    console.log(m, m.content);
    var url = null;
    try {                
        url = JSON.parse(m.content).id;
        if (typeof url != 'string') url = longUrl;
    } catch (e) {
        url = longUrl;
    }
	linkarea.innerHTML = "";			
	var textinput = document.createElement("INPUT");
	textinput.type = "text";
	textinput.disabled = true;
	textinput.value = url;

	var link = document.createElement("A");
	link.href = url;
	link.innerHTML = "Try lesson";

	linkarea.appendChild(textinput);
	linkarea.appendChild(link);
	textinput.select();
}

\end{verbatim}

Получает ответ в формате JSON и, распарсив, выводит на страницу  constructor.html в текстовое окно.
\begin{verbatim}
var getShortenURL = function(url) {
    jsonlib.fetch({
            url: 'https://www.googleapis.com/urlshortener/v1/url',
            header: 'Content-Type: application/json',
            method: 'POST',
            data: JSON.stringify({longUrl: url})
        }, 
        function(m){parseShortenedResponse(m, url)});
    }
\end{verbatim}

\vspace{16mm}
\textbf{\Large A2. Очистка экрана}
\vspace{6mm}

Очищает прямоугольник, размером с Canvas 
\begin{verbatim}
...
var clearCanvas;
Tortuga.initDrawing = function(canvas){
	...
    clearCanvas = function()	{
        ctx.clearRect(0, 0, canvas.width, canvas.height);
    }
}
\end{verbatim}

\vspace{16mm}
\textbf{\Large A3. Изменение толщины рисования}
\vspace{6mm}

В объект, представляющий черепаху, добавлено свойство width:
\begin{verbatim}
var Tortoise = function(xx, yy, color, width, tortoiseContainer)	{		
    ...
    this.width = width || 1;
    ...
}
\end{verbatim}
В прототип (объект, представляющий свойства и методы, общие для всех черепах) была добавлена функция setWidth:

\begin{verbatim}
var proto = {
    go : function(t, length){
        ...
        if(t.isDrawing){
        ...
        oldWidth = setWidth(t.width);
        ...
        setWidth(oldWidth);
        }
    },
    ...
    setWidth : function(t, w){t.width = w || t.width}
}
\end{verbatim}

Добавлена публичная функция setWidth для отражения на Canvas действий черепахи:

\begin{verbatim}
Tortuga.initDrawing = function(canvas){
    ...
    setWidth = function(width){
        ctx.lineWidth = width;
    }
}
\end{verbatim}

\vspace{16mm}
\textbf{\Large A4. Редактирование уроков}
\vspace{6mm}

Формирование необходимых элементов для ввода и вывода информации:
\begin{verbatim}
<input type="text" id="lessonInput">
<button id="changebutton">Изменить урок</button><br>
<textarea id="area"></textarea><br>
<button id="createbutton">Создать урок</button><br>
<div id="linkarea"></div>
\end{verbatim}

Передача данных из  текстового поля lessonInput в объект Черепаха:
\begin{verbatim}
<script>
var init = function(){
    ...
    Tortuga.givLessonArea(document.getElementById("area"),
        document.getElementById("changebutton"),
        document.getElementById("lessonInput"));
}
</script>
\end{verbatim}

Вызов закрытой функции updateArea:
\begin{verbatim}
Tortuga.givLessonArea = function(area, button, input){
    button.onclick = function(e){updateArea(area.value, input.value)}
}
\end{verbatim}

Обрабатывание полученных из lessonInput  данных:
\begin{verbatim}
var updateArea = function (areaValue, inputValue){
    var t =  Tortuga.ParamsUtil.getLessonTextFromGetUriValue(inputValue);
    var paramBegin = null;
    var paramAnd = null;
    var paramText = null;
    var resultText = "";

    paramBegin = t.indexOf(':"');
    paramAnd = t.indexOf('"', paramBegin + 2);

	
    while (paramBegin > 0){
        paramText = t.substr(paramBegin + 2, paramAnd - paramBegin - 2);
        resultText = resultText + paramText + '\n\n';
        t = t.substr(paramAnd + 2);
        paramBegin = t.indexOf(':"');
        paramAnd = t.indexOf('"', paramBegin + 2);
    }

    document.getElementById('area').value = resultText;
}
\end{verbatim}

Извлечение текста урока из ссылки:

\begin{verbatim}
var getUriValue = function(u){
    var str = u;
    var vhozhd = str.indexOf('?');
    var result = str.substr(vhozhd + 1);
    return result;
}
var getLessonTextFromGetUriValue = function(u){
    return prezip_to_utf8(RawDeflate.inflate(atob(getUriValue(u))));
}
\end{verbatim}

% % % % % % % % % % % % % % % % % % % % %
\vspace{16mm}
\textbf{\Large A6. Изменение урока \# без перезагрузки страницы}
\vspace{6mm}

описание действия
\begin{verbatim}

if ("onhashchange" in window)
    {
        window.onhashchange = function () 
        {
            replacementDOMListOfTabs(bg, list, descrDiv, env, lesson);
        }
    }
    else
    {
        var storedHash = window.location.hash;
        window.setInterval(function () 
            {
                if (window.location.hash != storedHash) 
                {
                    replacementDOMListOfTabs(bg, list, descrDiv, env, lesson);
                }
            }, 100);
    }

var buildListOfTabs = function(bg, list, descrDiv, env, lesson)
{
    list.appendChild(createList(lesson.items, bg, descrDiv,
    new LessonEnv(env, lesson.title))); 
}

var replacementDOMListOfTabs = function(bg, list, descrDiv, env, lesson)
{
    for(var i=1; i<=list.children.length; i++) 
    {
        var child = list.children[i];
        list.removeChild(child);
    }

    buildListOfTabs(bg, list, descrDiv, env, lesson);
}

\end{verbatim}

% % % % % % % % % % % % % % % % % % % % %
\vspace{16mm}
\textbf{\Large A8. Методы изменения координат угла и поворота черепашки}
\vspace{6mm}

\textbf{TORTOISE}
\begin{verbatim}
var setCoords = function(jsConverter, jsTortoise, x, y)
    {
        if (typeof x == "number")
        {
            jsConverter.parseNode(jsConverter.nodes.setX, jsTortoise, x)
            jsConverter.parseNode(jsConverter.nodes.setY, jsTortoise, y)
        } else 
        {
            jsConverter.parseNode(jsConverter.nodes.setX, jsTortoise, x.x)
            jsConverter.parseNode(jsConverter.nodes.setY, jsTortoise, x.y)
        }	
    }

var getAngle = function(jsConverter, jsTortoise)
    {
    return jsConverter.parseNode(jsConverter.nodes.getAngle, jsTortoise).value
    }
\end{verbatim}

Добавлены в конструктор комманд
\begin{verbatim}
var constructProto = function(jsConverter)
    {
        return {
            .........
            getAngle: jsConverter.nodes.getAngle,
            setAngle: jsConverter.nodes.setAngle,
        }
    }
\end{verbatim}



добавлены в свойства черепахи
\begin{verbatim}
Tortoise.prototype.getAngle = function()
        {
            return getAngle(jsConverter, this.jsTortoise)
        }
\end{verbatim}


\textbf{JS CONVERTER}
добавлены  ноды
\begin{verbatim}
........
    var NODE_GET_ANGLE  = new FirstParamIsVariableResultNode(TR.commands.getAngle)
    var NODE_SET_ANGLE  = new FirstParamIsVariableNode(TR.commands.setAngle)
    var NODE_SET_X      = new FirstParamIsVariableNode(TR.commands.setX)
    var NODE_SET_Y      = new FirstParamIsVariableNode(TR.commands.setY)

JsConverter.prototype.nodes = {
        ........
        setX     : NODE_SET_X,
        setY     : NODE_SET_Y,
        getAngle : 	NODE_GET_ANGLE,
        setAngle :  NODE_SET_ANGLE,
    }
\end{verbatim}


\textbf{JS CONVERTER}
\begin{verbatim}
var runGetAngle = function runGetAngle(runner, getTrTortoise, handler)
    {
        handler(getTrTortoise().deg)
    }

var runSetAngle = function runGetAngle(runner, getTrTortoise, deg)
    {
        getTrTortoise().deg = deg
    }
var runSetX = function runSetX(runner, getTrTortoise, x)
    {
        getTrTortoise().x = x
    }
var runSetY = function runSetX(runner, getTrTortoise, y)
    {
        getTrTortoise().y = y
    }
TortoiseRunner.commands = {
        .......
        setX     : runSetX,
        setY     : runSetY,
        getAngle : runGetAngle,
        setAngle : runSetAngle,
        .......
    }
\end{verbatim}





% % % % % % % % % % % % % % % % % % % % %
\vspace{16mm}
\textbf{\Large A9. Обработка всех мышинных событий в рамках канвы}
\vspace{6mm}

\textbf{DRAWINGSISTEM}

добавлены методы возвращающие координаты черепахи в координатах canvas, и наоборот
\begin{verbatim}
var convertCoordsTortugaToCanvas = function(ctx, tortugaX, tortugaY)
    {
        return {x : tortugaX, y : (ctx.canvas.height - tortugaY)}
    }

var convertCoordsCanvasToTortuga = function(ctx, canvasX, canvasY)
    {
        return {x : canvasX, y : (ctx.canvas.height - canvasY)}
    }
\end{verbatim}

\textbf{MOUSEJS}
\begin{verbatim}
ns("Tortuga.Events");
(function()
{
    var tortugaEventsHolder = Tortuga.Events;

    Tortuga.initMouse = function(drawingSystem) 
    {
        var handlerEvent = function(e)
        {
            var point = drawingSystem.convertCoordsCanvasToTortuga(e.layerX, e.layerY)
            var event_canvas = 
                {
                    tortugaX: point.x,
                    tortugaY: point.y,
                    originalEvent: e
                }

            var event_name = "on" + e.type
            if (typeof tortugaEventsHolder[event_name] == "function")
            {
                tortugaEventsHolder[event_name](event_canvas)
            }
        }

        var canvas = drawingSystem.getCanvas()
        var events = [
            "onclick", "onmousedown", "onmouseenter", "onmouseleave",
            "onmousemove", "onmouseout", "onmouseover", "onmouseup",
            "onmousewheel"
        ]
        events.forEach(function(event_name)
        {
            canvas[event_name] = handlerEvent
            tortugaEventsHolder[event_name] = null
        })
    }
})()



\end{verbatim}


% % % % % % % % % % % % % % % % % % % % %
\vspace{16mm}
\textbf{\Large A10. D\&D файлов}
\vspace{6mm}

\textbf{FILES.JS}
\begin{verbatim}
var processScript =  function(script)
    {
        preAction();
        var scriptElement = document.createElement("script");
        scriptElement.innerHTML = script;

        var headElement = document.getElementsByTagName("head")[0];
        headElement.appendChild(scriptElement);
        postAction();
    }

var processFile = function(file)
    {
        var reader = new FileReader();
        reader.onload = function(e)
        {
            processScript(e.target.result)
        };

        reader.readAsText(file);
    }


var processText = function(script)
    {
        processScript(script)
    }

var preventDefault = function(e)
    {
        e.preventDefault ? e.preventDefault() : (e.returnValue = false);
    }


var doDrop = function(e) {
    canvasObjekt.classList.remove("tortuga-canvasContainer-dragging")
        if (e.dataTransfer.files)
        {
            var file = e.dataTransfer.files;
            for (var i=0; i<file.length; i++)
            {
                processFile(file[i]);
            }
            preventDefault(e)
        }

        if (e.dataTransfer.getData('Text'))
        {
            var text = e.dataTransfer.getData('Text');
            processText(text);
            preventDefault(e)
        }
        return false;
    }

var handleDragOver = function(e)
    {
        preventDefault(e)
        canvasObjekt.classList.add("tortuga-canvasContainer-dragging")
        console.log("rrr");
    }
var handleDragLeave = function()
    {
        canvasObjekt.classList.remove("tortuga-canvasContainer-dragging")
    }


...
canvasObjekt.addEventListener("dragleave", handleDragLeave, false);
canvasObjekt.addEventListener("dragover", handleDragOver, false);
canvasObjekt.addEventListener("drop", dodrop, false);

    }
\end{verbatim}