\appendix
\chapter{} 

\textbf{\Large A1. Сокращение ссылки} \label{AppendixA3}
\vspace{6mm}

Отправляет запрос с передачей длинной ссылки:
\begin{verbatim}
var parseShortenedResponse = function(m, longUrl) {
    console.log(m, m.content);
    var url = null;
    try {                
        url = JSON.parse(m.content).id;
        if (typeof url != 'string') url = longUrl;
    } catch (e) {
        url = longUrl;
    }
	linkarea.innerHTML = "";			
	var textinput = document.createElement("INPUT");
	textinput.type = "text";
	textinput.disabled = true;
	textinput.value = url;

	var link = document.createElement("A");
	link.href = url;
	link.innerHTML = "Try lesson";

	linkarea.appendChild(textinput);
	linkarea.appendChild(link);
	textinput.select();
}

\end{verbatim}

Получает ответ в формате JSON и, распарсив, выводит на страницу  constructor.html в текстовое окно.
\begin{verbatim}
var getShortenURL = function(url) {
    jsonlib.fetch({
            url: 'https://www.googleapis.com/urlshortener/v1/url',
            header: 'Content-Type: application/json',
            method: 'POST',
            data: JSON.stringify({longUrl: url})
        }, 
        function(m){parseShortenedResponse(m, url)});
    }
\end{verbatim}

\vspace{16mm}
\textbf{\Large A2. Очистка экрана}
\vspace{6mm}

Очищает прямоугольник, размером с Canvas 
\begin{verbatim}
...
var clearCanvas;
Tortuga.initDrawing = function(canvas){
	...
    clearCanvas = function()	{
        ctx.clearRect(0, 0, canvas.width, canvas.height);
    }
}
\end{verbatim}

\vspace{16mm}
\textbf{\Large A3. Изменение толщины рисования}
\vspace{6mm}

В объект, представляющий черепаху, добавлено свойство width:
\begin{verbatim}
var Tortoise = function(xx, yy, color, width, tortoiseContainer)	{		
    ...
    this.width = width || 1;
    ...
}
\end{verbatim}
В прототип (объект, представляющий свойства и методы, общие для всех черепах) была добавлена функция setWidth:

\begin{verbatim}
var proto = {
    go : function(t, length){
        ...
        if(t.isDrawing){
        ...
        oldWidth = setWidth(t.width);
        ...
        setWidth(oldWidth);
        }
    },
    ...
    setWidth : function(t, w){t.width = w || t.width}
}
\end{verbatim}

Добавлена публичная функция setWidth для отражения на Canvas действий черепахи:

\begin{verbatim}
Tortuga.initDrawing = function(canvas){
    ...
    setWidth = function(width){
        ctx.lineWidth = width;
    }
}
\end{verbatim}

\vspace{16mm}
\textbf{\Large A4. Редактирование уроков}
\vspace{6mm}

Формирование необходимых элементов для ввода и вывода информации:
\begin{verbatim}
<input type="text" id="lessonInput">
<button id="changebutton">Изменить урок</button><br>
<textarea id="area"></textarea><br>
<button id="createbutton">Создать урок</button><br>
<div id="linkarea"></div>
\end{verbatim}

Передача данных из  текстового поля lessonInput в объект Черепаха:
\begin{verbatim}
<script>
var init = function(){
    ...
    Tortuga.givLessonArea(document.getElementById("area"),
        document.getElementById("changebutton"),
        document.getElementById("lessonInput"));
}
</script>
\end{verbatim}

Вызов закрытой функции updateArea:
\begin{verbatim}
Tortuga.givLessonArea = function(area, button, input){
    button.onclick = function(e){updateArea(area.value, input.value)}
}
\end{verbatim}

Обрабатывание полученных из lessonInput  данных:
\begin{verbatim}
var updateArea = function (areaValue, inputValue){
    var t =  Tortuga.ParamsUtil.getLessonTextFromGetUriValue(inputValue);
    var paramBegin = null;
    var paramAnd = null;
    var paramText = null;
    var resultText = "";

    paramBegin = t.indexOf(':"');
    paramAnd = t.indexOf('"', paramBegin + 2);

	
    while (paramBegin > 0){
        paramText = t.substr(paramBegin + 2, paramAnd - paramBegin - 2);
        resultText = resultText + paramText + '\n\n';
        t = t.substr(paramAnd + 2);
        paramBegin = t.indexOf(':"');
        paramAnd = t.indexOf('"', paramBegin + 2);
    }

    document.getElementById('area').value = resultText;
}
\end{verbatim}

Извлечение текста урока из ссылки:

\begin{verbatim}
var getUriValue = function(u){
    var str = u;
    var vhozhd = str.indexOf('?');
    var result = str.substr(vhozhd + 1);
    return result;
}
var getLessonTextFromGetUriValue = function(u){
    return prezip_to_utf8(RawDeflate.inflate(atob(getUriValue(u))));
}
\end{verbatim}

% % % % % % % % % % % % % % % % % % % % %

